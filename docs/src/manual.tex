\documentclass[a4paper,12pt]{article}
\usepackage{style}
\usepackage{cite}

\begin{document}
    \docNumber{RU.17701729.04.06-01 33 01-1}
    \docFormat{Руководство программиста}
    \student{БПИ 199}{Д.А. Щербаков}
    \project{Модуль для мобильных приложений для определения эмоционального отклика по изображению пользователя.}
    \supervisor{Аналитик-разработчик \par АО <<Тинькофф банк>>}
    {Весельев А.Н.}
    \firstPage
    \newpage
    \annotation
    \section*{Аннотация}
    Данный документ содержит руководство программиста к программному модулю <<Facial Expression Recognition Mobile Library>> (<<Модуль для мобильных приложений для определения эмоционального отклика по изображению пользователя>>).
    Модуль служит для определения эмоций человека по изображению его лица, в том числе в режиме реального времени (например, по изображению с камеры).
    В разделе <<Назначение и условия применения программного модуля>> указаны назначение и функции, выполняемые программным модулем, условия, необходимые для работы программы.
    В разделе <<Характеристики программного модуля>> приведено описание основных характеристик и особенностей программного модуля.
    В разделе <<Обращение к программному модулю>> приведено описание процедур вызова программного модуля.
    В разделе <<Входные и выходные данные>> приведено описание организации используемой входной и выходной информации.

    Настоящий документ разработан в соответствии с требованиями:
    \begin{enumerate}
        \item ГОСТ 19.101-77 Виды программ и программных документов~\cite{gost1};
        \item ГОСТ 19.103-77 Обозначения программ и программных документов~\cite{gost2};
        \item ГОСТ 19.102-77 Стадии разработки~\cite{gost3};
        \item ГОСТ 19.104-78 Основные надписи~\cite{gost4};
        \item ГОСТ 19.105-78 Общие требования к программным документам~\cite{gost5};
        \item ГОСТ 19.106-78 Требования к программным документам, выполненным печатным способом~\cite{gost6};
        \item ГОСТ 19.504-79 Руководство программиста. Требования к содержанию и оформлению~\cite{gost8};
    \end{enumerate}

    \newpage

    \thirdPage

    \newpage

    \section{Назначение и условия применения программного модуля}
    \subsection{Функциональное назначение}
    Модуль позволяет определить эмоции человека на изображении по заданной шкале эмоций.

    Прилагаемый к модулю пример эксплуатации позволяет определять эмоциональный отклик на различные записи в социальной сети <<Reddit>> с помощью данных с камеры пользователя.

    \subsection{Эксплуатационное назначение}
    Модуль и пример использования должны эксплуатироваться на смартфонах под управлением операционной системы Android.

    \subsection{Область применения}
    Программа может быть использована в мобильных приложениях для Android в качестве аналитического модуля для улучшения пользовательского опыта или сбора информации о предпочтениях пользователя.

    \subsection{Состав технических и программных средств}
    Для работы программного модуля необходим следующий набор программных средств:
    \begin{enumerate}
        \item операционная система Android версии 8.0 и выше
    \end{enumerate}

    Для работы программного модуля необходим следующий состав технических средств:
    \begin{enumerate}
        \item Не менее 512МБ ОЗУ;
        \item Не менее 150МБ свободного места на внутреннем накопителе;
    \end{enumerate}

    Для анализа изображения с камеры необходимо наличие на устройстве фронтальной камеры.

    \newpage

    \section{Характеристики программного модуля}

    Данный программный модуль оформлен в виде библиотеки для приложений для платформы Android версии 8.0 и выше.

    Анализ изображения работает на основе нейронных сетей с помощью библиотеки MLKit и PyTorch.
    Код библиотеки написан на языке JAVA версии стандарта 8 и использует сторонние библиотеки, написанные на Kotlin и C++.

    Модуль, подключенный к программе на устройстве предоставляет набор функций для:
    \begin{itemize}
        \item Определения эмоций человека по передаваемому изображению любого размера.
        \item Определения эмоций человека в разные моменты времени по записанному видеопотоку.
        \item Определения эмоций человека в текущий момент времени с помощью изображения с камеры на устройстве пользователя.
    \end{itemize}

    \newpage

    \section{Обращение к программному модулю}

    Подключение модуля к программе для платформы Android происходит с помощью системы сборки Gradle.

    Для подключения необходимо выполнить следующие шаги:
    \begin{enumerate}
        \item Загрузить исходный код программного модуля на компьютер, на котором разрабатывается приложение
        \item В файле сборки build.gradle подключить библиотеку с помощью команды dependencies, передав в качестве аргумента строчку
            \begin{verbatim}
implementation project(path: ':facialExpressionRecognitionLib')
            \end{verbatim}
            \vskip -1.5em
            Необходимо заменить facialExpressionRecognitionLib на путь до папки с библиотекой
        \item Обновить проект gradle
    \end{enumerate}

    После этого можно обращаться к программному модулю, используя предоставляемые им программные интерфейсы, приведенные в пояснительной записке к данному модулю.

    Основной метод для анализа изображений -- \texttt{CameraFeedAnalyzer::analyze}, в качестве параметра которому передается
    объект типа \texttt{ImageProxy}, являющийся оберткой над классом \texttt{Image}, содержащей изображение.

    Для анализа изображения с камеры необходимо подключить в графический интерфейс приложения фрагмент \texttt{CameraFeedView}:

    \begin{verbatim}
<androidx.fragment.app.FragmentContainerView
        android:id="@+id/cameraFeedView"
        android:name="facialExpressionRecognition.CameraFeedView"
        android:layout_height="150dp" />
    \end{verbatim}
    \vskip -1.5em
    Данное действие необходимо, так как камера не сможет работать без отображения вывода на экране пользователя
    по соображениям безопасности операционной системы Android.

    Взаимодействие с UI-фрагментом происходит с помощью класса \texttt{FacialExpression-ViewModel}.
    Его интерфейс позволяет в режиме реального времени извлекать результат анализа изображения,
    а также подписываться на изменения для реагирования на них при изменении показателей
    (например, когда изменилось выражение лица пользователя и модуль прочитал новую эмоцию).

    \newpage

    \section{Входные и выходные данные}

    Ввод данных для анализа (изображение) может производиться непосредственно с помощью вызова метода
    \texttt{CameraFeedAnalyzer::analyze}, в качестве входного параметра которого передается изображение в
    формате YUV_420_888 и обернутое в класс \texttt{ImageProxy}. \texttt{ImageProxy} позволяет проводить анализ
    в режиме реального времени, так как различные библиотеки для работы с потоком изображений из камеры или из видео
    предоставляют методы для асинхронного подключения обработчиков изображений с помощью этого интерфейса.

    Если используется непосредственно модуль для работы с камерой, то извлечение данных происходит автоматически,
    и программисту не нужно ничего передавать.

    Класс \texttt{FacialExpressionViewModel} предоставляет несколько методов для извлечения информации,
    полученной классом \texttt{CameraFeedAnalyzer}.
    Программист может получить такие данные, как:

    \begin{itemize}
        \item Координаты границ лица на картинке в виде прямоугольника, вмещающего в себя лицо (метод \texttt{getFaceRect}).
        \item Название одной из семи эмоций, которая наиболее вероятно изображена на лице пользователя (метод \texttt{getLabel}).
        \item Массив из семи вещественных чисел, в котором каждое число соответствует вероятности того, что на лице пользователя
            изображена конкретная эмоция (метод \texttt{getResult}).
            Метод полезно использовать для анализа смешанных эмоций.
        \item Массив целых чисел типа int размера $44\times44$, в котором каждое число обозначает цвет пикселя того изображения,
            которое передается в нейросеть (метод \texttt{getFaceImage}).
            Метод используется для отладки правильности преобразования изображения.
    \end{itemize}

    Все данные методы возвращают объекты типа \texttt{LiveData}, из которых можно получить лежащий в них объект сразу,
    или подписаться на изменения с помощью метода \texttt{observe}, передав в него функцию для наблюдения.

    \newpage

    \renewcommand{\refname}{Список источников}
    \addcontentsline{toc}{section}{\refname}
    \bibliographystyle{ieeetr}
    \bibliography{bibliography}
\end{document}