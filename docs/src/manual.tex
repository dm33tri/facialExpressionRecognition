\documentclass[a4paper,12pt]{article}
\usepackage{style}
\usepackage{cite}

\begin{document}
    \docNumber{RU.17701729.04.06-01 33 01-1}
    \docFormat{Руководство программиста}
    \student{БПИ 199}{Д.А. Щербаков}
    \project{Модуль для мобильных приложений для определения эмоционального отклика по изображению пользователя.}
    \supervisor{Аналитик-разработчик \par АО <<Тинькофф банк>>}
    {Весельев А.Н.}
    \firstPage
    \newpage
    \annotation
    \section*{Аннотация}
    Данный документ содержит руководство программиста к программному модулю <<Facial Expression Recognition Mobile Library>> (<<Модуль для мобильных приложений для определения эмоционального отклика по изображению пользователя>>).
    Модуль служит для определения эмоций человека по изображению его лица, в том числе в режиме реального времени (например, по изображению с камеры).
    В разделе <<Назначение и условия применения программного модуля>> указаны назначение и функции, выполняемые программным модулем, условия, необходимые для работы программы.
    В разделе <<Характеристики программного модуля>> приведено описание основных характеристик и особенностей программного модуля.
    В разделе <<Обращение к программному модулю>> приведено описание процедур вызова программного модуля.
    В разделе <<Входные и выходные данные>> приведено описание организации используемой входной и выходной информации.

    Настоящий документ разработан в соответствии с требованиями:
    \begin{enumerate}
        \item ГОСТ 19.101-77 Виды программ и программных документов~\cite{gost1};
        \item ГОСТ 19.103-77 Обозначения программ и программных документов~\cite{gost2};
        \item ГОСТ 19.102-77 Стадии разработки~\cite{gost3};
        \item ГОСТ 19.104-78 Основные надписи~\cite{gost4};
        \item ГОСТ 19.105-78 Общие требования к программным документам~\cite{gost5};
        \item ГОСТ 19.106-78 Требования к программным документам, выполненным печатным способом~\cite{gost6};
        \item ГОСТ 19.504-79 Руководство программиста. Требования к содержанию и оформлению~\cite{gost8};
    \end{enumerate}

    \newpage

    \thirdPage

    \newpage

    \section{Назначение и условия применения программного модуля}

    \newpage

    \section{Характеристики программного модуля}

    \newpage

    \section{Обращение к программному модулю}

    \newpage

    \section{Входные и выходные данные}

    \newpage

    \section{Сообщения}

    \newpage
    \renewcommand{\refname}{Список источников}
    \addcontentsline{toc}{section}{\refname}
    \bibliographystyle{ieeetr}
    \bibliography{bibliography}
\end{document}