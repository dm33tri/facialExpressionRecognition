\documentclass[a4paper,12pt,reqno]{article}

\usepackage{style}

\begin{document}
    \docNumber{RU.17701729.04.06-01 81 ТЗ 01-1}
    \docFormat{Техническое задание}
    \student{БПИ 199}{Д.А. Щербаков}
    \project{Модуль для мобильных приложений для определения эмоционального отклика по изображению пользователя.}
    \supervisor{Аналитик-разработчик \par АО <<Тинькофф банк>>}
    {Весельев А. Н.}

    \firstPage

    \newpage
    \secondPage

    \newpage
    \thirdPage

    \newpage
    \section{Введение}
    \subsection{Наименование программы}
    \subsubsection{Наименование темы разработки на русском языке}
    Модуль для мобильных приложений для определения эмоционального отклика по изображению пользователя.
    \subsubsection{Наименование темы разработки на английском языке}
    Module for mobile applications for reading emotional response from a user's image.
    \subsection{Краткая характеристика области применения}
    Программный модуль может быть использован в любых мобильных приложениях для анализа эмоционального отклика пользователей в разные моменты времени.

    \newpage
    \section{Основания для разработки}
    \subsection{Документы, на основании которых ведется разработка}
    Приказ декана факультета компьютерных наук Национального Исследовательского университета «Высшая школа экономики» от 2020 г. № X.X-XX/XXXX-XX «Об утверждении тем, руководителей выпускных квалификационных работ студентов образовательной программы Программная инженерия факультета компьютерных наук».
    \subsection{Наименование темы разработки}
    Наименование темы разработки – <<Модуль для мобильных приложений для определения эмоционального отклика по изображению пользователя>>.

    \newpage
    \section{Назначение разработки}
    \subsection{Функциональное назначение}
    Модуль позволяет определить эмоции человека на изображении по заданной шкале эмоций. Прилагаемый к модулю пример эксплуатации позволяет определять эмоциональный отклик на различные записи в социальной сети <<Reddit>> с помощью данных с камеры пользователя.
    \subsection{Эксплуатационное назначение}
    Модуль и пример использования должны эксплуатироваться на смартфонах под управлением операционной системы Android.

    \newpage
    \section{Требования к программе}
    \subsection{Требования к функциональным характеристикам}
    \subsubsection{Требования к составу выполняемых функций}
    Программный модуль должен:
    \begin{itemize}
        \item Предоставлять программный интерфейс для определения эмоций человека по изображению, переданному этому модулю из программы.
        \item Предоставлять программный интерфейс для определения эмоций человека в реальном времени по видеопотоку, передаваемому из программы.
        \item Предоставлять программный интерфейс для считывания эмоций пользователя с помощью камеры устройства.
    \end{itemize}
    Программа-пример эксплуатации должна:
    \begin{itemize}
        \item Отображать на экране пользователя случайно выбранные картинки из социальной сети <<Reddit>> в виде бесконечной ленты с возможностью переключаться между картинками жестами на сенсорном экране.
        \item Измерять эмоциональный отклик пользователя на картинки по изображению с фронтальной камеры устройства и сохранять данные в локальную базу данных.
        \item Отображать результаты измерений в отдельном разделе в виде просмотренных записей и подписей с полученным по каждой записи результатом.
    \end{itemize}
    \subsubsection{Требования к организации входных данных}
    Модуль, подключенный к программе на устройстве, должен предоставлять набор функций для
    \begin{itemize}
        \item Определения эмоций человека по передаваемому изображению любого размера.
        \item Определения эмоций человека в разные моменты времени по записанному видеопотоку.
        \item Определения эмоций человека в текущий момент времени с помощью изображения с камеры на устройстве пользователя.
    \end{itemize}
    \subsubsection{Требования к организации выходных данных}
    Предоставляемые функции должны возвращать ассоциативный массив с ключами в виде закодированного обозначения конкретной эмоции (грустный, веселый, удивленный и т.п.) и значениями в виде вещественных чисел от 0 до 1, обозначающими степень выраженности соответственной эмоции. Если не удалось распознать эмоции, например, в случае, если на изображении нет человека или их несколько, то передавать соответствующий код ошибки.
    \subsubsection{Требования к надёжности}
    При любом вводе модуль не должен вызывать аварийное завершение программы, в которую он включен.

    \subsection{Требования к интерфейсу}
    \subsubsection{Требования к программному интерфейсу}
    Модуль должен предоставлять следующий интерфейс для его использования в других программах:
    \begin{itemize}
        \item Модуль должен быть выполнен в виде библиотеки, которую можно статически (на этапе компиляции) включить в программу для операционной системы Android.
        \item Модуль должен предоставлять для программиста набор функций, которые при вызове с правильными параметрами возвращают результат измерений в один момент времени или множество результатов за определенный промежуток времени, если в качестве данных мы передали видеопоток.
        \item При передаче неправильных параметров или при невозможности измерений функция должна возвращать соответствующий код ошибки.
    \end{itemize}
    \subsubsection{Требования к пользовательскому интерфейсу}
    Программа-пример эксплуатации должна иметь следующий пользовательский интерфейс:
    \begin{itemize}
        \item Основное окно - открывается при запуске приложения:
        \begin{itemize}
            \item Бесконечная лента из изображений с подписями к ним.
            \item В одно время отображается только одно изображение.
            \item Менять изображения можно с помощью жеста <<свайп вверх>> на сенсорном экране.
        \end{itemize}
        \item Кнопка для открытия окна с результатами измерений.
        \item Окно с результатами измерений:
        \begin{itemize}
            \item Список изображений с подписями в 2 ряда и несколько строчек с вертикальной прокруткой с помощью жестов на сенсорном экране.
            \item Под каждым изображением список обнаруженных эмоций и степень их выраженности в процентах.
        \end{itemize}
    \end{itemize}

    \subsection{Условия эксплуатации}
    \subsubsection{Климатические условия}
    Климатические условия сопадают с климатическими условиями эксплуатации устройства.
    \subsubsection{Требования к пользователю}
    Пользователь должен иметь базовое представление об основных принципах работы с пользовательским интерфейсом операционной системы Android.
    \subsection{Требования к составу и параметру технических средств}
    Минимальные требования программы для работоспособности на смартфоне:
    \begin{itemize}
        \item Процессор с частотой не менее 1 ГГц
        \item Сенсорный экран с разрешением не менее 320 точек по одному из измерений;
        \item Не менее 512МБ ОЗУ;
        \item Не менее 128МБ на внутреннем накопителе;
    \end{itemize}
    \subsection{Требования к информационной и программной совместимости}
    \subsubsection{Требования к программынм средствам, используемым программой}
    \begin{itemize}
        \item На смартфоне должна быть установлена операционная система Android версии 9.0 или выше.
    \end{itemize}
    \subsubsection{Требования к исходным кодам и языкам программирования}
    \begin{itemize}
        \item Модель для распознавания эмоций должна быть написана на языке Python и использовать фреймворк PyTorch.
        \item Программный модуль должен быть написан на языке Kotlin.
        \item Программа-пример эксплуатации должна быть написана на языке Kotlin.
    \end{itemize}

    \subsection{Требования к маркировке и упаковке}
    Программный модуль должен быть доступен для скачивания в виде исходных кодов и готовой модели для включения в программу на этапе компиляции.

    Программа-пример эксплуатации должна быть доступна для скачивания в виде установщика .apk для Android.

    \newpage
    \section{Требования к программной документации}
    В рамках данной работы должна быть разработана следующая программная документация в соответствии и ГОСТ ЕСПД:
    \begin{itemize}
        \item «Модуль для мобильных приложений для определения эмоционального отклика по изображению пользователя». Техническое задание \cite{TZ};
        \item «Модуль для мобильных приложений для определения эмоционального отклика по изображению пользователя». Программа и методика испытаний \cite{TEST};
        \item «Модуль для мобильных приложений для определения эмоционального отклика по изображению пользователя». Текст программы \cite{TEXT};
        \item «Модуль для мобильных приложений для определения эмоционального отклика по изображению пользователя». Пояснительная записка \cite{NOTE};
        \item «Модуль для мобильных приложений для определения эмоционального отклика по изображению пользователя». Руководство оператора \cite{MANUAL};
        \item «Модуль для мобильных приложений для определения эмоционального отклика по изображению пользователя». Руководство программиста \cite{PROG};
    \end{itemize}

    \newpage
    \section{Технико-экономические показатели}
    \subsection{Предполагаемая потребность}
    Программный модуль будет использовать в других программах для определения эмоционального отклика пользователей на различные события и материалы и дальнейшего анализа полученных данных в целях улучшения пользовательского опыта.

    \subsection{Экономические преимущества разработки по сравнению с отечественными и зарубежными аналогами}
    Существуют различные модели для анализа эмоций пользователя, реализованные с помощью фреймворков PyTorch или Tensorflow, но все они не предоставляют удобный программный интерфейс для интеграции в программы и использования на устройствах пользователя. Также разрабатываемая модель предполагает быстрый анализ видеопотока -- гораздо быстрее, чем существующие модели, которые обрабатывают одно изображение с большой задержкой и повышенным использованием вычислительных ресурсов.

    \newpage
    \section{Стадии и этапы разработки}
    \subsection{Техническое задание}
    Обоснование необходимости разработки
    \begin{itemize}
        \item Постановка задачи;
        \item Сбор теоретического материала;
        \item Выбор и обоснование критериев эффективности и качества разрабатываемого продукта.
    \end{itemize}

    Научно-исследовательские работы
    \begin{itemize}
        \item Предварительный выбор методов решения поставленной задачи;
        \item Определение требований к техническим средствам;
        \item Обоснование возможности решения поставленной задачи.
    \end{itemize}

    Разработка и утверждение технического задания
    \begin{itemize}
        \item Определение требований к программе;
        \item Определение стадий, этапов и сроков разработки программы и документации на неё;
        \item Выбор языка программирования;
        \item Согласование и утверждение технического задания.
    \end{itemize}
    \subsection{Рабочий проект}
    Разработка программы
    \begin{itemize}
        \item Реализация протокола и алгоритма совместного редактирования;
        \item Реализация программного интерфейса;
        \item Отладка программы.
    \end{itemize}

    Разработка программной документации
    \begin{itemize}
        \item Разработка программных документов в соответствии с требованиями ЕСПД.
    \end{itemize}

    Испытания программы
    \begin{itemize}
        \item Разработка, согласование и утверждение программы и методики испытаний;
        \item Проведение предварительных приемо-сдаточных испытаний;
        \item Корректировка программы и программной документации по результатам испытаний.
    \end{itemize}

    \subsection{Внедрение}
    Подготовка и защита программного продукта
    \begin{itemize}
        \item Подготовка программы и документации для защиты;
        \item Утверждение дня защиты программы;
        \item Презентация разработанного программного продукта;
        \item Передача программы и программной документации в архив НИУ ВШЭ.
    \end{itemize}

    \newpage
    \patchcmd{\thebibliography}{\section*{\refname}}{}{}{}
    \section{Список источников}
    \begin{thebibliography}{3}
        \bibitem{TZ} ГОСТ 19.201-78 Техническое задание. Требования к содержанию и оформлению. //Единая система программной документации. – М.: ИПК Издательство стандартов, 2001.

        \bibitem{PC} ГОСТ 15150-69 Машины, приборы и другие технические изделия. Исполнения для различных климатических районов. Категории, условия эксплуатации, хранения и транспортирования в части воздействия климатических факторов внешней среды. – М.: Изд-во стандартов, 1997.

        \bibitem{TEST} ГОСТ 19.301-79 Программа и методика испытаний. Требования к содержанию и оформлению. //Единая система программной документации. – М.: ИПК Издательство стандартов, 2001.

        \bibitem{TEXT} ГОСТ 19.401-78. ЕСПД. Текст программы. Требования к содержанию и оформлению. – М.: ИПК Издательство стандартов, 2001.

        \bibitem{NOTE} ГОСТ 19.404-79. ЕСПД. Пояснительная записка. Требования к содержанию и оформлению. – М.: ИПК Издательство стандартов, 2001.

        \bibitem{MANUAL} ГОСТ 19.505-79. ЕСПД. Руководство оператора. Требования к содержанию и оформлению. – М.: ИПК Издательство стандартов, 2001.

        \bibitem{PROG} ГОСТ 19.504-79. ЕСПД. Руководство программиста. Требования к содержанию и оформлению. – М.: ИПК Издательство стандартов, 2001.
    \end{thebibliography}

    \newpage
    \section*{Лист регистрации изменений}
    \listRegistration

\end{document}