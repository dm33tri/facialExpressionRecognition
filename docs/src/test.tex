\documentclass[a4paper,12pt]{article}
\usepackage{style}
\usepackage{cite}

\begin{document}
    \docNumber{RU.17701729.04.06-01 34 01-1}
    \docFormat{Программа и методика испытаний}
    \student{БПИ 199}{Д.А. Щербаков}
    \project{Модуль для мобильных приложений для определения эмоционального отклика по изображению пользователя.}
    \supervisor{Аналитик-разработчик \par АО <<Тинькофф банк>>}
    {Весельев А.Н.}
    \firstPage
    \newpage
    \annotation
    \section*{Аннотация}
    Данный документ содержит программу и методику испытаний к программному модулю <<Facial Expression Recognition Mobile Library>> (<<Модуль для мобильных приложений для определения эмоционального отклика по изображению пользователя>>).
    Модуль служит для определения эмоций человека по изображению его лица, в том числе в режиме реального времени (например, по изображению с камеры).
    В разделе <<Объект испытаний>> указано наименование, область применения и обозначение объекта испытаний.
    В разделе <<Цель испытаний>> указана цель проведения испытаний.
    В разделе <<Требования к программе>> указаны требования, подлежащие проверке во время испытаний и заданные в техническом задании на программу.
    В разделе <<Требования к программной документации>> указаны состав программной документации, предъявляемой на испытания, а также специальные требования, если они заданы в техническом задании на программу.
    В разделе <<Средства и порядок испытаний>> указаны технические и программные средства, используемые во время испытаний, а также порядок проведения испытаний.
    В разделе <<Методы испытаний>> приведены описания используемых методов испытаний.

    Настоящий документ разработан в соответствии с требованиями:
    \begin{enumerate}
        \item ГОСТ 19.101-77 Виды программ и программных документов~\cite{gost1};
        \item ГОСТ 19.103-77 Обозначения программ и программных документов~\cite{gost2};
        \item ГОСТ 19.102-77 Стадии разработки~\cite{gost3};
        \item ГОСТ 19.104-78 Основные надписи~\cite{gost4};
        \item ГОСТ 19.105-78 Общие требования к программным документам~\cite{gost5};
        \item ГОСТ 19.106-78 Требования к программным документам, выполненным печатным способом~\cite{gost6};
        \item ГОСТ 19.301-79 Программа и методика испытаний. Требования к содержанию и оформлению~\cite{gost9};
    \end{enumerate}

    \newpage

    \thirdPage

    \newpage

    \section{Объект испытаний}

    \newpage

    \section{Цель испытаний}
    Проверка программного модуля <<facialExpressionRecognitionLib> на соответствие требованиям, указанным в техническом задании.
    \newpage

    \section{Требования к программному модулю}

    \subsubsection{Требования к составу выполняемых функций}
    Программный модуль должен:
    \begin{itemize}
        \item Предоставлять программный интерфейс для определения эмоций человека по изображению, переданному этому модулю из программы.
        \item Предоставлять программный интерфейс для определения эмоций человека в реальном времени по видеопотоку, передаваемому из программы.
        \item Предоставлять программный интерфейс для считывания эмоций пользователя с помощью камеры устройства.
    \end{itemize}
    \subsubsection{Требования к организации входных данных}
    Модуль, подключенный к программе на устройстве, должен предоставлять набор функций для
    \begin{itemize}
        \item Определения эмоций человека по передаваемому изображению любого размера.
        \item Определения эмоций человека в разные моменты времени по записанному видеопотоку.
        \item Определения эмоций человека в текущий момент времени с помощью изображения с камеры на устройстве пользователя.
    \end{itemize}
    \subsubsection{Требования к организации выходных данных}
    Предоставляемые функции должны возвращать ассоциативный массив с ключами в виде закодированного обозначения конкретной эмоции (грустный, веселый, удивленный и т.п.) и значениями в виде вещественных чисел от 0 до 1, обозначающими степень выраженности соответственной эмоции. Если не удалось распознать эмоции, например, в случае, если на изображении нет человека или их несколько, то передавать соответствующий код ошибки.
    \subsubsection{Требования к надёжности}
    При любом вводе модуль не должен вызывать аварийное завершение программы, в которую он включен.

    \newpage

    \section{Требования к программной документации}
    Состав программной документации включает в себя:
    \begin{enumerate}
        \item <<Модуль для мобильных приложений для определения эмоционального отклика по изображению пользователя>>. Техническое задание;
        \item <<Модуль для мобильных приложений для определения эмоционального отклика по изображению пользователя>>. Руководство оператора;
        \item <<Модуль для мобильных приложений для определения эмоционального отклика по изображению пользователя>>. Программа и методика испытаний;
        \item <<Модуль для мобильных приложений для определения эмоционального отклика по изображению пользователя>>. Текст программы;
    \end{enumerate}

    \newpage

    \section{Средства и порядок испытаний}
    \subsection{Технические средства}
    \begin{itemize}
        \item Не менее 512МБ ОЗУ;
        \item Не менее 150МБ свободного места на внутреннем накопителе;
    \end{itemize}

    Для анализа изображения с камеры необходимо наличие на устройстве фронтальной камеры.

    \subsection{Требования к информационной совместимости}
    Для работы программного модуля необходим следующий набор программных средств:
    \begin{itemize}
        \item операционная система Android версии 8.0 и выше
    \end{itemize}
    
    \newpage

    \section{Методы испытаний}
    \begin{enumerate}
        \item Установка программы
        \item Запуск программы
        \item Определение корректности выделения лица
        \item Определение корректности определения эмоций
    \end{enumerate}

    \newpage
    \renewcommand{\refname}{Список источников}
    \addcontentsline{toc}{section}{\refname}
    \bibliographystyle{ieeetr}
    \bibliography{bibliography}
\end{document}