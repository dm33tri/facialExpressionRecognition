\documentclass[11pt,a4paper]{report}
\usepackage{color}
\usepackage{ifthen}
\usepackage{makeidx}
\usepackage{ifpdf}
\usepackage[headings]{fullpage}
\usepackage{listings}
\lstset{language=Java,breaklines=true}
\ifpdf \usepackage[pdftex, pdfpagemode={UseOutlines},bookmarks,colorlinks,linkcolor={blue},plainpages=false,pdfpagelabels,citecolor={red},breaklinks=true]{hyperref}
  \usepackage[pdftex]{graphicx}
  \pdfcompresslevel=9
  \DeclareGraphicsRule{*}{mps}{*}{}
\else
  \usepackage[dvips]{graphicx}
\fi

\newcommand{\entityintro}[3]{%
  \hbox to \hsize{%
    \vbox{%
      \hbox to .2in{}%
    }%
    {\bf  #1}%
    \dotfill\pageref{#2}%
  }
  \makebox[\hsize]{%
    \parbox{.4in}{}%
    \parbox[l]{5in}{%
      \vspace{1mm}%
      #3%
      \vspace{1mm}%
    }%
  }%
}
\newcommand{\refdefined}[1]{
\expandafter\ifx\csname r@#1\endcsname\relax
\relax\else
{$($in \ref{#1}, page \pageref{#1}$)$}\fi}
\date{\today}
\chardef\textbackslash=`\\
\makeindex
\begin{document}
\sloppy
\addtocontents{toc}{\protect\markboth{Contents}{Contents}}
\tableofcontents
\chapter{Пакет ru.hse.dascherbakov\_1.facialExpressionRecognition}{
\label{ru.hse.dascherbakov_1.facialExpressionRecognition}\hskip -.05in
\vskip .13in
\hbox{{\bf  Классы}}
\entityintro{CameraFeedAnalyzer}{ru.hse.dascherbakov_1.facialExpressionRecognition.CameraFeedAnalyzer}{Класс анализатора изображения}
\entityintro{CameraFeedView}{ru.hse.dascherbakov_1.facialExpressionRecognition.CameraFeedView}{Фрагмент для предпросмотра изображения с камеры и анализа эмоций по изображению лица в реальном времени}
\entityintro{FacialExpressionViewModel}{ru.hse.dascherbakov_1.facialExpressionRecognition.FacialExpressionViewModel}{}
\vskip .1in
\vskip .1in
\section{\label{ru.hse.dascherbakov_1.facialExpressionRecognition.CameraFeedAnalyzer}\index{CameraFeedAnalyzer}Class CameraFeedAnalyzer}{
\vskip .1in 
Класс анализатора изображения\vskip .1in 
\subsection{Определение}{
\begin{lstlisting}[frame=none]
public class CameraFeedAnalyzer
 extends java.lang.Object\end{lstlisting}
\subsection{Field summary}{
\begin{verse}
{\bf classes} \\
\end{verse}
}
\subsection{Краткое описание конструкторов}{
\begin{verse}
{\bf CameraFeedAnalyzer(CameraFeedView)} \\
\end{verse}
}
\subsection{Краткое описание методов}{
\begin{verse}
{\bf analyze(ImageProxy)} Основной метод для анализа\\
{\bf buildAnalysis(CameraFeedView)} Получить анализатор для подключения к камере\\
\end{verse}
}
\subsection{Fields}{
\begin{itemize}
\item{
\index{classes}
\label{ru.hse.dascherbakov_1.facialExpressionRecognition.CameraFeedAnalyzer.classes}\texttt{public static final java.lang.String\lbrack \rbrack \ {\bf  classes}}
}
\end{itemize}
}
\subsection{Конструкторы}{
\vskip -2em
\begin{itemize}
\item{ 
\index{CameraFeedAnalyzer(CameraFeedView)}
{\bf  CameraFeedAnalyzer}\\
\begin{lstlisting}[frame=none]
public CameraFeedAnalyzer(CameraFeedView cameraFeedView)\end{lstlisting} %end signature
}%end item
\end{itemize}
}
\subsection{Методы}{
\vskip -2em
\begin{itemize}
\item{ 
\index{analyze(ImageProxy)}
{\bf  analyze}\\
\begin{lstlisting}[frame=none]
public void analyze(ImageProxy imageProxy)\end{lstlisting} %end signature
\begin{itemize}
\item{
{\bf  Описание}

Основной метод для анализа
}
\end{itemize}
}%end item
\item{ 
\index{buildAnalysis(CameraFeedView)}
{\bf  buildAnalysis}\\
\begin{lstlisting}[frame=none]
public static ImageAnalysis buildAnalysis(CameraFeedView cameraFeedView)\end{lstlisting} %end signature
\begin{itemize}
\item{
{\bf  Описание}

Получить анализатор для подключения к камере
}
\end{itemize}
}%end item
\end{itemize}
}
}
\section{\label{ru.hse.dascherbakov_1.facialExpressionRecognition.CameraFeedView}\index{CameraFeedView}Class CameraFeedView}{
\vskip .1in 
Фрагмент для предпросмотра изображения с камеры и анализа эмоций по изображению лица в реальном времени\vskip .1in 
\subsection{Определение}{
\begin{lstlisting}[frame=none]
public class CameraFeedView
 extends Fragment\end{lstlisting}
\subsection{Краткое описание конструкторов}{
\begin{verse}
{\bf CameraFeedView()} \\
\end{verse}
}
\subsection{Краткое описание методов}{
\begin{verse}
{\bf getViewModel()} Получение класса ViewModel\\
{\bf onActivityCreated(Bundle)} Завершение создания фрагмента\\
{\bf onCreateView(LayoutInflater, ViewGroup, Bundle)} Инициализация визуальных виджетов фрагмента\\
{\bf onViewCreated(View, Bundle)} Метод, выполняемый после инициализации визуальных виджетов фрагмента\\
{\bf setOnClickListener(View.OnClickListener)} \\
\end{verse}
}
\subsection{Конструкторы}{
\vskip -2em
\begin{itemize}
\item{ 
\index{CameraFeedView()}
{\bf  CameraFeedView}\\
\begin{lstlisting}[frame=none]
public CameraFeedView()\end{lstlisting} %end signature
}%end item
\end{itemize}
}
\subsection{Методы}{
\vskip -2em
\begin{itemize}
\item{ 
\index{getViewModel()}
{\bf  getViewModel}\\
\begin{lstlisting}[frame=none]
public FacialExpressionViewModel getViewModel()\end{lstlisting} %end signature
\begin{itemize}
\item{
{\bf  Описание}

Получение класса ViewModel
}
\end{itemize}
}%end item
\item{ 
\index{onActivityCreated(Bundle)}
{\bf  onActivityCreated}\\
\begin{lstlisting}[frame=none]
public void onActivityCreated(Bundle savedInstance)\end{lstlisting} %end signature
\begin{itemize}
\item{
{\bf  Описание}

Завершение создания фрагмента
}
\end{itemize}
}%end item
\item{ 
\index{onCreateView(LayoutInflater, ViewGroup, Bundle)}
{\bf  onCreateView}\\
\begin{lstlisting}[frame=none]
public View onCreateView(LayoutInflater inflater,ViewGroup viewGroup,Bundle bundle)\end{lstlisting} %end signature
\begin{itemize}
\item{
{\bf  Описание}

Инициализация визуальных виджетов фрагмента
}
\end{itemize}
}%end item
\item{ 
\index{onViewCreated(View, Bundle)}
{\bf  onViewCreated}\\
\begin{lstlisting}[frame=none]
public void onViewCreated(View view,Bundle bundle)\end{lstlisting} %end signature
\begin{itemize}
\item{
{\bf  Описание}

Метод, выполняемый после инициализации визуальных виджетов фрагмента
}
\end{itemize}
}%end item
\item{ 
\index{setOnClickListener(View.OnClickListener)}
{\bf  setOnClickListener}\\
\begin{lstlisting}[frame=none]
public void setOnClickListener(View.OnClickListener listener)\end{lstlisting} %end signature
}%end item
\end{itemize}
}
}
\section{\label{ru.hse.dascherbakov_1.facialExpressionRecognition.FacialExpressionViewModel}\index{FacialExpressionViewModel}Class FacialExpressionViewModel}{
\vskip .1in 
\subsection{Определение}{
\begin{lstlisting}[frame=none]
public class FacialExpressionViewModel
 extends ViewModel\end{lstlisting}
\subsection{Краткое описание конструкторов}{
\begin{verse}
{\bf FacialExpressionViewModel()} \\
\end{verse}
}
\subsection{Краткое описание методов}{
\begin{verse}
{\bf getFaceRect()} Получить прямоугольник границ лица\\
{\bf getImageSize()} Получить размер изображения\\
{\bf getLabel()} Получить название эмоции\\
{\bf getOutput()} Получить вектор с вероятностями эмоций\\
{\bf setFaceRect(RectF)} Записать прямоугольник границ лица\\
{\bf setImageSize(Size)} Записать размер изображения\\
{\bf setLabel(String)} Записать название эмоции\\
{\bf setOutput(float\lbrack \rbrack )} Записать вектор с вероятностями эмоций\\
\end{verse}
}
\subsection{Конструкторы}{
\vskip -2em
\begin{itemize}
\item{ 
\index{FacialExpressionViewModel()}
{\bf  FacialExpressionViewModel}\\
\begin{lstlisting}[frame=none]
public FacialExpressionViewModel()\end{lstlisting} %end signature
}%end item
\end{itemize}
}
\subsection{Методы}{
\vskip -2em
\begin{itemize}
\item{ 
\index{getFaceRect()}
{\bf  getFaceRect}\\
\begin{lstlisting}[frame=none]
public <any> getFaceRect()\end{lstlisting} %end signature
\begin{itemize}
\item{
{\bf  Описание}

Получить прямоугольник границ лица
}
\end{itemize}
}%end item
\item{ 
\index{getImageSize()}
{\bf  getImageSize}\\
\begin{lstlisting}[frame=none]
public <any> getImageSize()\end{lstlisting} %end signature
\begin{itemize}
\item{
{\bf  Описание}

Получить размер изображения
}
\end{itemize}
}%end item
\item{ 
\index{getLabel()}
{\bf  getLabel}\\
\begin{lstlisting}[frame=none]
public <any> getLabel()\end{lstlisting} %end signature
\begin{itemize}
\item{
{\bf  Описание}

Получить название эмоции
}
\end{itemize}
}%end item
\item{ 
\index{getOutput()}
{\bf  getOutput}\\
\begin{lstlisting}[frame=none]
public <any> getOutput()\end{lstlisting} %end signature
\begin{itemize}
\item{
{\bf  Описание}

Получить вектор с вероятностями эмоций
}
\end{itemize}
}%end item
\item{ 
\index{setFaceRect(RectF)}
{\bf  setFaceRect}\\
\begin{lstlisting}[frame=none]
public void setFaceRect(RectF faceRect)\end{lstlisting} %end signature
\begin{itemize}
\item{
{\bf  Описание}

Записать прямоугольник границ лица
}
\end{itemize}
}%end item
\item{ 
\index{setImageSize(Size)}
{\bf  setImageSize}\\
\begin{lstlisting}[frame=none]
public void setImageSize(Size imageSize)\end{lstlisting} %end signature
\begin{itemize}
\item{
{\bf  Описание}

Записать размер изображения
}
\end{itemize}
}%end item
\item{ 
\index{setLabel(String)}
{\bf  setLabel}\\
\begin{lstlisting}[frame=none]
public void setLabel(java.lang.String label)\end{lstlisting} %end signature
\begin{itemize}
\item{
{\bf  Описание}

Записать название эмоции
}
\end{itemize}
}%end item
\item{ 
\index{setOutput(float\lbrack \rbrack )}
{\bf  setOutput}\\
\begin{lstlisting}[frame=none]
public void setOutput(float[] output)\end{lstlisting} %end signature
\begin{itemize}
\item{
{\bf  Описание}

Записать вектор с вероятностями эмоций
}
\end{itemize}
}%end item
\end{itemize}
}
}
}
\printindex
\end{document}
